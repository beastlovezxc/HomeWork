%!TeX program = xelatex
%!TeX encoding = utf-8
\documentclass{ctexart}
	\usepackage{fourier}
	\usepackage{amsmath}
	\usepackage{anysize}
	\usepackage[level]{datetime}
	\marginsize{2cm}{2cm}{2cm}{2cm}
	\newdateformat{ukdate}{\ordinaldate{\THEDAY} \monthname[\THEMONTH] \THEYEAR}%日月年
	\title{The Design and Analysis of Algorithms's homework}
	\author{BeanCb}%陈斌  学号31617019
	\date{\ukdate\today}

	\begin{document}
		\maketitle
		\paragraph{3.1-2 Show that for any real constants $a$ and $b$, where $b > 0$,}
			\begin{equation}
				(n+a)^b=\Theta(n^b)\tag{3.2}
			\end{equation}
			\paragraph*{}
			Answer:

			To show $(n+a)^b=\Theta(n^b)$, first we find constants $c1$, $c2$, $n_0 > 0$ so we can get :
				\begin{equation}
					0\leqslant c_1n^b\leqslant (n+a)^b \leqslant c_2n^b\notag
				\end{equation}
				for all $n \leqslant n_0$.

				Note that
				\begin{equation}
					n + a \leqslant n + \left|a\right| \leqslant 2n\ when\ \left|a\right| \leqslant n\tag{3.2-1}
				\end{equation}

				and
				\begin{equation}
					n + a \geqslant n - \left|a\right| \geqslant \frac{1}{2}n\ when\ \left|a\right| \leqslant \frac{1}{2}n\tag{3.2-2}
				\end{equation}

				when $n \geqslant 2\left|a\right|$,
				\begin{equation}
					0 \leqslant \frac{1}{2}n \leqslant n + a \leqslant 2n.\tag{3.2-3}
				\end{equation}
				Since $b > 0$, it will still holds when all parts are raised to the b:
				\begin{equation}
					0 \leqslant (\frac{1}{2}n)^b \leqslant (n + a)^b \leqslant (2n)^b,\tag{3.2-4}
				\end{equation}
				\begin{equation}
					0 \leqslant (\frac{1}{2})^{b}n^b \leqslant (n + a)^b \leqslant 2^{b}n^b.\tag{3.2-5}
				\end{equation}
				So $c_1 = (\frac{1}{2})^b$, $c_2 = 2^b$, and $n_0 = 2\left|a\right|$ satisfy the difinition.
		\paragraph{3.1-3 Explain why the statement,"The running time of algorithm A is at least $O(n^2)$," is meaningless.}
			\paragraph*{}
			Answer:

			The running time of algorithm A is $T(n)$. $T(n) \geqslant O(n^2)$ means $T(n) \geqslant f(n)$ for some function $f(n)$ in the set $O(n^2)$.We get an upper bound for the worst situation to be the lower bound of the algorithm.So we know nothing about the running time.
		\paragraph{3.1-4 Is $2^{n+1} = O(2^n)$? Is $2^{2n} = O(2^n)$?}
			\paragraph*{}
			Answer:
			\begin{equation}
				2^{n+1} = O(2^n),\ 2^{2n} \not= O(2^n)\tag{3.4-1}
			\end{equation}
			To show $2^{n+1} = O(2^n)$,we must find constant $c$, $n_0 > 0$ so we can get 
			\begin{equation}
				0 \leqslant 2^{n+1} \leqslant c\cdot 2^n\ for\ all\ n \geqslant n_0.\tag{3.4-2}
			\end{equation}		
			Both side divide $2^n$ so we can get 
			\begin{equation}
				0 \leqslant 2 \leqslant c\tag{3.4-3}
			\end{equation}
			So we can satisfy the definition with $c \geqslant 2$ and $n_0 \geqslant 1$.

			To show $2^{2n} \not=O(2^n)$, we assume there exist constants $c$, $n_0 > 0$ so we can get 
			\begin{equation}
				0 \leqslant 2^{2n} \leqslant c\cdot2^n\ for\ all\ n \leqslant n_0\tag{3.4-4}
			\end{equation}
			Then both side divide $2^n$ so we can get
			\begin{equation}
				0 \leqslant 2^n \leqslant c\tag{3.4-5}
			\end{equation}
			So we can get $c \geqslant 2^n$ to satisfy the definition, but no constant is greater than all $2^n$,so the assumption does't hold.
	\end{document}